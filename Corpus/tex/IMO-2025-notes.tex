% © Evan Chen
% Downloaded from https://web.evanchen.cc/

\documentclass[11pt]{scrartcl}
\usepackage[sexy]{evan}
\ihead{\footnotesize\textbf{\thetitle}}
\ohead{\footnotesize\href{http://web.evanchen.cc}{\ttfamily web.evanchen.cc},
    updated \today}
\title{IMO 2025 Solution Notes}
\date{\today}

\begin{document}

\maketitle

\begin{abstract}
This is a compilation of solutions
for the 2025 IMO.
The ideas of the solution are a mix of my own work,
the solutions provided by the competition organizers,
and solutions found by the community.
However, all the writing is maintained by me.

These notes will tend to be a bit more advanced and terse than the ``official''
solutions from the organizers.
In particular, if a theorem or technique is not known to beginners
but is still considered ``standard'', then I often prefer to
use this theory anyways, rather than try to work around or conceal it.
For example, in geometry problems I typically use directed angles
without further comment, rather than awkwardly work around configuration issues.
Similarly, sentences like ``let $\mathbb{R}$ denote the set of real numbers''
are typically omitted entirely.

Corrections and comments are welcome!
\end{abstract}

\tableofcontents
\newpage

\addtocounter{section}{-1}
\section{Problems}
\begin{enumerate}[\bfseries 1.]
\item %% Problem 1
A line in the plane is called \emph{sunny}
if it is not parallel to any of the $x$–axis, the $y$–axis, or the line $x+y=0$.

Let $n \ge 3$ be a given integer.
Determine all nonnegative integers $k$ such that there exist $n$ distinct lines
in the plane satisfying both of the following:
\begin{itemize}
\ii for all positive integers $a$ and $b$ with $a+b\le n+1$,
  the point $(a,b)$ lies on at least one of the lines; and
\ii exactly $k$ of the $n$ lines are sunny.
\end{itemize}

\item %% Problem 2
Let $\Omega$ and $\Gamma$ be circles with centres $M$ and $N$, respectively,
such that the radius of $\Omega$ is less than the radius of $\Gamma$.
Suppose $\Omega$ and $\Gamma$ intersect at two distinct points $A$ and $B$.
Line $MN$ intersects $\Omega$ at $C$ and $\Gamma$ at $D$,
so that $C$, $M$, $N$, $D$ lie on $MN$ in that order.
Let $P$ be the circumcenter of triangle $ACD$.
Line $AP$ meets $\Omega$ again at $E \neq A$ and meets $\Gamma$ again at $F \neq A$.
Let $H$ be the orthocenter of triangle $PMN$.

Prove that the line through $H$ parallel to $AP$ is tangent
to the circumcircle of triangle $BEF$.

\item %% Problem 3
A function $f \colon \NN \to \NN$ is said to be \emph{bonza} if
\[ f(a)\quad\text{divides}\quad b^a-f(b)^{f(a)} \]
for all positive integers $a$ and $b$.

Determine the smallest real constant $c$ such that $f(n) \leq cn$
for all bonza functions $f$ and all positive integers $n$.

\item %% Problem 4
An infinite sequence $a_1$, $a_2$, \dots\ consists of positive integers
has each of which has at least three proper divisors.
Suppose that for each $n\geq 1$,
$a_{n+1}$ is the sum of the three largest proper divisors of $a_n$.
Determine all possible values of $a_1$.

\item %% Problem 5
Alice and Bazza are playing the \emph{inekoalaty game},
a two‑player game whose rules depend on a positive real number $\lambda$ which is known to both players.
On the $n$th turn of the game (starting with $n=1$) the following happens:
\begin{itemize}
  \ii If $n$ is odd, Alice chooses a nonnegative real number $x_n$ such that
  \[ x_1 + x_2 + \cdots + x_n \le \lambda n.  \]
  \ii If $n$ is even, Bazza chooses a nonnegative real number $x_n$ such that
  \[ x_1^2 + x_2^2 + \cdots + x_n^2 \le n.  \]
\end{itemize}
If a player cannot choose a suitable $x_n$, the game ends and the other player wins.
If the game goes on forever, neither player wins.
All chosen numbers are known to both players.

Determine all values of $\lambda$ for which Alice has a winning strategy
and all those for which Bazza has a winning strategy.

\item %% Problem 6
Consider a $2025 \times 2025$ grid of unit squares.
Matilda wishes to place on the grid some rectangular tiles, possibly of different sizes,
such that each side of every tile lies on a grid line
and every unit square is covered by at most one tile.

Determine the minimum number of tiles Matilda needs to place
so that each row and each column of the grid has exactly one unit square
that is not covered by any tile.

\end{enumerate}
\pagebreak

\section{Solutions to Day 1}
\subsection{IMO 2025/1, proposed by Linus Tang (USA)}
\textsl{Available online at \url{https://aops.com/community/p35332003}.}
\begin{mdframed}[style=mdpurplebox,frametitle={Problem statement}]
A line in the plane is called \emph{sunny}
if it is not parallel to any of the $x$–axis, the $y$–axis, or the line $x+y=0$.

Let $n \ge 3$ be a given integer.
Determine all nonnegative integers $k$ such that there exist $n$ distinct lines
in the plane satisfying both of the following:
\begin{itemize}
\ii for all positive integers $a$ and $b$ with $a+b\le n+1$,
  the point $(a,b)$ lies on at least one of the lines; and
\ii exactly $k$ of the $n$ lines are sunny.
\end{itemize}
\end{mdframed}
The answer is $0$, $1$, or $3$ sunny lines.

In what follows, we draw the grid as equilateral instead of a right triangle;
this has no effect on the problem statement but is more symmetric.

We say a \emph{long line} is one of the three lines at the edge of the grid,
i.e.\ one of the (non-sunny) lines passing through $n$ points.
The main claim is the following.
\begin{claim*}
  If $n \ge 4$, any set of $n$ lines must have at least one long line.
\end{claim*}
\begin{proof}
  Consider the $3(n-1)$ points on the outer edge of the grid.
  If there was no long line, each of the $n$ lines passes through at most two such points.
  So we obtain $2n \ge 3(n-1)$, which forces $n \le 3$.
\end{proof}

Hence, by induction we may repeatedly delete a long line without changing the number
of sunny lines until $n = 3$ (and vice-versa: given a construction for smaller $n$
we can increase $n$ by one and add a long line).

We now classify \emph{all} the ways to cover
the $1+2+3=6$ points in an $n=3$ grid with $3$ lines.
\begin{center}
\begin{asy}
  dotfactor *= 2;
  unitsize(2cm);
  pair A = dir(90);
  pair B = dir(210);
  pair C = dir(330);
  pair X = midpoint(B--C);
  pair Y = midpoint(C--A);
  pair Z = midpoint(A--B);
  picture tri;
  dot(tri, A);
  dot(tri, B);
  dot(tri, C);
  dot(tri, X);
  dot(tri, Y);
  dot(tri, Z);
  picture left_case;
  picture right_case;
  add(left_case, tri);
  add(right_case, tri);

  draw(left_case, (1.2*A-0.2*B)--(1.2*B-0.2*A), red, Arrows);
  draw(left_case, circle((C+X+Y)/3, 0.75), blue);
  label(left_case, "Long line present", 1.2*dir(-90));

  draw(right_case, (1.3*A-0.3*X)--(1.3*X-0.3*A), deepgreen, Arrows);
  draw(right_case, (1.3*B-0.3*Y)--(1.3*Y-0.3*B), deepgreen, Arrows);
  draw(right_case, (1.3*C-0.3*Z)--(1.3*Z-0.3*C), deepgreen, Arrows);

  label(right_case, "No long line", 1.2*dir(-90));

  add(left_case);
  add(shift(3,0)*right_case);
\end{asy}
\end{center}

\begin{itemize}
  \ii If there is a long line
  (say, the red one in the figure),
  the remaining $1+2=3$ points (circled in blue) are covered with two lines.
  One of the lines passes through $2$ points and must not be sunny;
  the other line may or may not be sunny.
  Hence in this case the possible counts of sunny lines are $0$ or $1$.

  \ii If there is no long line, each of the three lines passes through at most $2$ points.
  But there are $6$ total lines, so in fact each line must pass through
  \emph{exactly} two points.
  The only way to do this is depicted in the figure in the right.
  In this case there are $3$ sunny lines.
\end{itemize}

This proves that $0$, $1$, $3$ are the only possible answers.

\begin{remark*}
  The concept of a sunny line is not that important to the problem.
  The proof above essentially classifies all the ways to cover
  the $1+2+\dots+n$ points with exactly $n$ lines.
  Namely, one should repeatedly take a long line and decrease $n$ until $n=3$,
  and then pick one of the finitely many cases for $n=3$.
  The count of sunny lines just happens to be whatever is possible for $n=3$,
  since long lines are not sunny.
\end{remark*}
\pagebreak

\subsection{IMO 2025/2, proposed by Tran Quang Hung (VNM)}
\textsl{Available online at \url{https://aops.com/community/p35332018}.}
\begin{mdframed}[style=mdpurplebox,frametitle={Problem statement}]
Let $\Omega$ and $\Gamma$ be circles with centres $M$ and $N$, respectively,
such that the radius of $\Omega$ is less than the radius of $\Gamma$.
Suppose $\Omega$ and $\Gamma$ intersect at two distinct points $A$ and $B$.
Line $MN$ intersects $\Omega$ at $C$ and $\Gamma$ at $D$,
so that $C$, $M$, $N$, $D$ lie on $MN$ in that order.
Let $P$ be the circumcenter of triangle $ACD$.
Line $AP$ meets $\Omega$ again at $E \neq A$ and meets $\Gamma$ again at $F \neq A$.
Let $H$ be the orthocenter of triangle $PMN$.

Prove that the line through $H$ parallel to $AP$ is tangent
to the circumcircle of triangle $BEF$.
\end{mdframed}
Throughout the solution, we define
\begin{align*}
  \alpha &\coloneq \dang DCA = \dang BCD \implies \dang PAD = \dang CAB = 90\dg - \alpha \\
  \beta &\coloneq \dang ADC = \dang CDB \implies \dang CAP = \dang BAD = 90\dg - \beta.
\end{align*}
Ignore the points $H$, $M$, $N$ for now and focus on the remaining ones.
\begin{claim*}
  We have $\ol{CE} \parallel \ol{AD}$ and $\ol{DF} \parallel \ol{AC}$.
\end{claim*}
\begin{proof}
  $\dang AEC = \dang ABC = \dang CAB = 90\dg - \alpha$.
\end{proof}
Hence, if we let $A' \coloneq \ol{CE} \cap \ol{DF}$, we have a parallelogram $ACA'D$.
Note in particular that $\ol{BA'} \parallel \ol{CD}$.

\begin{center}
\begin{asy}
size(15cm);

pair A = dir(117);
pair C = dir(150);
pair D = dir(30);
filldraw(unitcircle, opacity(0.1)+lightcyan, blue);
pair B = -A+2*foot(A, C, D);
pair M = circumcenter(A, B, C);
pair N = circumcenter(A, B, D);
pair P = origin;
draw(circumcircle(A, C, B), gray);
draw(circumcircle(A, D, B), gray);
draw(C--D, blue);
draw(A--B, blue);
pair H = orthocenter(P, M, N);
pair A_prime = C+D-A;
pair T = M+N-A;
pair E = extension(A, P, M, T);
pair F = extension(A, P, N, T);
draw(A--F, lightred);
draw(C--A_prime, lightred);
draw(F--D, lightred);
draw(M--H--N, lightred);
draw(circumcircle(A_prime, E, F), gray);

draw(A_prime--B, blue);
draw(circumcircle(B, E, F), purple);
draw(M--T, dashed);
draw(N--F, dashed);
draw(C--A--D, lightred);
markangle("$\alpha$", n=1, radius=18.0, D,C,A, deepgreen);
markangle("$\beta$", n=2, radius=30.0, A,D,C, deepgreen);
markangle("$\alpha$", n=1, radius=18.0, C,D,F, deepgreen);
markangle("$\beta$", n=2, radius=30.0, E,C,D, deepgreen);
markangle("$90^{\circ}-\alpha$", n=3, radius=14.0, C,A,B, deepcyan);
markangle("$90^{\circ}-\alpha$", n=3, radius=14.0, E,A,D, deepcyan);

dot("$A$", A, dir(A));
dot("$C$", C, dir(C));
dot("$D$", D, dir(D));
dot("$B$", B, dir(225));
dot("$M$", M, dir(125));
dot("$N$", N, dir(N));
dot("$P$", P, dir(250));
dot("$H$", H, dir(270));
dot("$A'$", A_prime, dir(315));
dot("$T$", T, dir(230));
dot("$E$", E, dir(260));
dot("$F$", F, dir(315));

/* --------------------------------+
| TSQX: by CJ Quines and Evan Chen |
| https://github.com/vEnhance/tsqx |
+----------------------------------+
!size(15cm);

A = dir 117
C = dir 150
D = dir 30
unitcircle / 0.1 lightcyan / blue
B 225 = -A+2*foot A C D
M 125 = circumcenter A B C
N = circumcenter A B D
P 250 = origin
circumcircle A C B / gray
circumcircle A D B / gray
C--D / blue
A--B / blue
H 270 = orthocenter P M N
A' 315 = C+D-A
T 230 = M+N-A
E 260 = extension A P M T
F 315 = extension A P N T
A--F / lightred
C--A' / lightred
F--D / lightred
M--H--N / lightred
circumcircle A' E F / gray

A'--B / blue
circumcircle B E F / purple
M--T / dashed
N--F / dashed
C--A--D / lightred
!markangle("$\alpha$", n=1, radius=18.0, D,C,A, deepgreen);
!markangle("$\beta$", n=2, radius=30.0, A,D,C, deepgreen);
!markangle("$\alpha$", n=1, radius=18.0, C,D,F, deepgreen);
!markangle("$\beta$", n=2, radius=30.0, E,C,D, deepgreen);
!markangle("$90^{\circ}-\alpha$", n=3, radius=14.0, C,A,B, deepcyan);
!markangle("$90^{\circ}-\alpha$", n=3, radius=14.0, E,A,D, deepcyan);
*/
\end{asy}
\end{center}


Next, let $T$ denote the circumcenter of $\triangle A'EF$.
(This will be the tangency point later in the problem.)
\begin{claim*}
  Point $T$ also lies on $\ol{BA'}$ and is also the arc midpoint of $\widehat{EF}$ on $(BEF)$.
\end{claim*}
\begin{proof}
  We compute the angles of $\triangle A'EF$:
  \begin{align*}
    \dang FEA' &= \dang AEC = \dang ABC = \dang CAB = 90\dg - \alpha \\
    \dang A'FE &= \dang DFA = \dang DBA = \dang BAD = 90\dg - \beta \\
    \dang EA'F &= \alpha + \beta.
  \end{align*}
  Then, since $T$ is the circumcenter, it follows that:
  \[ \dang EA'T = \dang 90\dg - \dang A'FE = \beta = \dang A'CD = \dang CA'B. \]
  This shows that $T$ lies on $\ol{BA'}$.

  Also, we have $\dang ETF = 2 \dang EA'F = 2(\alpha+\beta)$ and
  \begin{align*}
    \dang EBF &= \dang EBA + \dang ABF = \dang ECA + \dang ADF \\
    &= \dang A'CA + \dang ADA' = (\alpha+\beta) + (\alpha+\beta) = 2(\alpha+\beta)
  \end{align*}
   which proves that $T$ also lies on $\ol{AB'}$.
\end{proof}

We then bring $M$ and $N$ into the picture as follows:
\begin{claim*}
  Point $T$ lies on both lines $ME$ and $NF$.
\end{claim*}
\begin{proof}
  To show that $F$, $T$, $N$ are collinear,
  note that $\triangle FEA' \sim \triangle FAD$ via a homothety at $F$.
  This homothety maps $T$ to $N$.
\end{proof}

We now deal with point $H$ using two claims.
\begin{claim*}
  We have $\ol{MH} \parallel \ol{AD}$ and $\ol{NH} \parallel \ol{AC}$.
\end{claim*}
\begin{proof}
  Note that $\ol{MH} \perp \ol{PN}$,
  but $\ol{PN}$ is the perpendicular bisector of $\ol{AD}$,
  so in fact $\ol{MH} \parallel \ol{AD}$.
  Similarly, $\ol{NH} \parallel \ol{AC}$.
\end{proof}
\begin{claim*}
  Lines $\ol{MH}$ and $\ol{NH}$ bisect $\angle NMT$ and $\angle MNT$.
  In fact, point $H$ is the incenter of $\triangle TMN$,
  and $\dang NTH = \dang HTM = 90\dg-(\alpha+\beta)$.
\end{claim*}
\begin{proof}
  Hence, $\dang HMN = \dang A'CD = \dang ADC = \beta$.
  But $\dang TMN = \dang CME = 2 \dang CAE = -2(90\dg-2\beta) = 2\beta$.
  That proves $\ol{MH}$ bisects $\angle NMT$; the other one is similar.

  To show that $H$ is an incenter (rather than an excenter)
  and get the last angle equality, we need to temporarily undirect our angles.
  Assume WLOG that $\triangle ACD$ is directed counterclockwise.
  The problem condition that $C$ and $D$ are the farther intersections of line $MN$
  mean that $\angle NHM = \angle CAD > 90\dg$.
  We are also promised $C$, $M$, $N$, $D$ are collinear in that order.
  Hence the reflections of line $MN$ over lines $MH$ and $NH$, which meet at $T$,
  should meet at a point for which $T$ lies on the same side as $H$.
  In other words, $\triangle MTN$ is oriented counterclockwise and contains $H$.

  Working with undirected $\alpha = \angle DCA$ and $\beta = \angle ADC$ with $\alpha + \beta < 90\dg$,
  \[ \angle NTH = \angle HTM = \half \angle NTM = \half(180\dg-2(\alpha+\beta))
    = 90\dg - (\alpha + \beta). \]
  This matches the claim and finishes the result.
\end{proof}

Now
\[ \dang NFA = 90\dg - \dang ADF = 90\dg-(\alpha+\beta) = \dang NTH \]
so $\ol{HT} \parallel \ol{AP}$.
And since $TE = TF$, we have the tangency requested too now, as desired.

\begin{remark*}
  There are many other ways to describe the point $T$.
  For example, $AMTN$ is a parallelogram and $MBTN$ is an isosceles trapezoid.
  In coordination, we joked that it was impossible to write a false conjecture.
\end{remark*}
\pagebreak

\subsection{IMO 2025/3, proposed by Lorenzo Sarria (COL)}
\textsl{Available online at \url{https://aops.com/community/p35332016}.}
\begin{mdframed}[style=mdpurplebox,frametitle={Problem statement}]
A function $f \colon \NN \to \NN$ is said to be \emph{bonza} if
\[ f(a)\quad\text{divides}\quad b^a-f(b)^{f(a)} \]
for all positive integers $a$ and $b$.

Determine the smallest real constant $c$ such that $f(n) \leq cn$
for all bonza functions $f$ and all positive integers $n$.
\end{mdframed}
The answer is $c=4$.

Let $P(a,b)$ denote the given statement $f(a) \mid b^a - f(b)^{f(a)}$.
\begin{claim*}
  We have $f(n) \mid n^n$ for all $n$.
\end{claim*}
\begin{proof}
  Take $P(n,n)$.
\end{proof}

\begin{claim*}
  Unless $f = \id$, we have $f(p) = 1$ for all odd primes $p$.
\end{claim*}
\begin{proof}
  Consider any prime $q$ with $f(q) > 1$.
  Then $f(q)$ is a power of $q$, and for each $n$ we get
  \[ P(q,n) \implies q \mid f(q) \mid n^q - f(n)^{f(q)}. \]
  Fermat's little theorem now gives
  $n^q \equiv n \pmod q$ and $f(n)^{f(q)} \equiv f(n) \pmod q$
  (since $f(q)$ is a power of $q$), and therefore $q \mid n - f(n)$.
  Hence, unless $f$ is the identity function,
  only finitely many $q$ could have $f(q) > 1$.

  Now let $p$ be any odd prime,
  and let $q$ be a large prime such that $q \not\equiv 1 \pmod p$
  (possible for all $p > 2$, say by Dirichlet).
  Then
  \[ P(p,q) \implies f(p) \mid q^p - 1^p. \]
  The RHS is $q^p - 1 \equiv q - 1 \not\equiv 0 \pmod p$, so $f(p) = 1$.
\end{proof}

\begin{claim*}
  We have $f(n) \mid 2^\infty$ for all $n$.
\end{claim*}
\begin{proof}
  If $p \mid f(n)$ is odd then $P(n,p)$ gives $p \mid f(n) \mid p^n-1^n$,
  contradiction.
\end{proof}
(In particular, we now know $f(n) = 1$ for all odd $n$, though we don't use this.)

\begin{claim*}
  We have $f(n) \le 2^{\nu_2(n)+2}$ for all $n$.
\end{claim*}
\begin{proof}
  Consider $P(n,5) \implies f(n) \mid 5^n - 1^n$.
  It's well-known that $\nu_2(5^n-1) = \nu_2(n)+2$ for all $n$.
\end{proof}
This immediately shows $f(n) \le 4n$ for all $n$, hence $c=4$ in the problem statement works.

For the construction, the simplest one seems to be
\[
  f(n) = \begin{cases}
    1 & n \text{ odd} \\
    16 & n = 4 \\
    2 & n \text{ even}, n \neq 4
  \end{cases}
\]
which is easily checked to work and has $f(4) = 16$.

\begin{remark*}
  With a little more case analysis we can classify all functions $f$.
  The two trivial solutions are $f(n) = n$ and $f(n) = 1$;
  the others are described by writing $f(n) = 2^{e(n)}$ for any function $e$ satisfying
  \begin{itemize}
    \ii $e(n) = 0$ for odd $n$;
    \ii $1 \le e(2) \le 2$;
    \ii $1 \le e(n) \le \nu_2(n)+2$ for even $n > 2$.
  \end{itemize}
  This basically means that there are almost no additional constraints
  beyond what is suggested by the latter two claims.
\end{remark*}
\pagebreak

\section{Solutions to Day 2}
\subsection{IMO 2025/4, proposed by Paulius Aleknavičius (LIT)}
\textsl{Available online at \url{https://aops.com/community/p35347364}.}
\begin{mdframed}[style=mdpurplebox,frametitle={Problem statement}]
An infinite sequence $a_1$, $a_2$, \dots\ consists of positive integers
has each of which has at least three proper divisors.
Suppose that for each $n\geq 1$,
$a_{n+1}$ is the sum of the three largest proper divisors of $a_n$.
Determine all possible values of $a_1$.
\end{mdframed}
The answer is $a_1 = 12^e \cdot 6 \cdot \ell$
for any $e, \ell \ge 0$ with $\gcd(\ell, 10) = 1$.

Let $\mathbf{S}$ denote the set of positive integers with at least three divisors.
For $x \in \mathbf{S}$, let $\psi(x)$ denote the sum of the three largest ones,
so that $\psi(a_n) = a_{n+1}$.

\paragraph{Proof that all such $a_1$ work.}
Let $x = 12^e \cdot 6 \cdot \ell \in \mathbf{S}$ with $\gcd(\ell, 10) = 1$.
As $\frac12+\frac13+\frac14 = \frac{13}{12}$, we get
\[ \psi(x) = \begin{cases}
    x & e = 0 \\
    \frac{13}{12} x & e > 0
  \end{cases} \]
so by induction on the value of $e$ we see that $\psi(x) \in \mathbf{S}$
(the base $e=0$ coming from $\psi$ fixing $x$).

\paragraph{Proof that all $a_1$ are of this form.}
In what follows $x$ is always an element of $\mathbf{S}$,
not necessarily an element of the sequence.

\begin{claim*}
  Let $x \in \mathbf{S}$. If $2 \mid \psi(x)$ then $2 \mid x$.
\end{claim*}
\begin{proof}
  If $x$ is odd then every divisor of $x$ is odd,
  so $f(x)$ is the sum of three odd numbers.
\end{proof}

\begin{claim*}
  Let $x \in \mathbf{S}$.  If $6 \mid \psi(x)$ then $6 \mid x$.
\end{claim*}
\begin{proof}
  We consider only $x$ even because of the previous claim.
  We prove the contrapositive that $3 \nmid x \implies 6 \nmid \psi(x)$ (for even $x$).
  \begin{itemize}
    \ii If $4 \mid x$, then letting $d$ be the third largest proper divisor of $x$,
    \[ \psi(x) = \frac x2 + \frac x4 + d = \frac 34 x + d \equiv d \not\equiv 0 \pmod 3. \]
    \ii Otherwise, let $p \mid x$ be the smallest prime dividing $x$, with $p > 3$.
    If the third-smallest nontrivial divisor of $x$ is $2p$, then
    \[ \psi(x) = \frac x2 + \frac xp + \frac{x}{2p} = \frac{3}{2p} x + \frac x2
      \equiv \frac x2 \not\equiv 0 \pmod 3. \]
    If the third-smallest nontrivial divisor of $x$ is instead an odd prime $q$, then
    \[ \psi(x) = \frac x2 + \frac xp + \frac{x}{q} \equiv 1+0+0 \equiv 1 \pmod 2. \qedhere \]
  \end{itemize}
\end{proof}

To tie these two claims into the problem, we assert:
\begin{claim*}
  Every $a_i$ must be divisible by $6$.
\end{claim*}
\begin{proof}
  The idea is to combine the previous two claims
  (which have no dependence on the sequence) with a size argument.
  \begin{itemize}
    \ii For odd $x \in \mathbf{S}$ note that $\psi(x) < \left( \frac13+\frac15+\frac17 \right) x < x$
    and $\psi(x)$ is still odd.
    So if any $a_i$ is odd the sequence is strictly decreasing and that's impossible.
    Hence, we may assume $a_1$, $a_2$, \dots\ are all even.

    \ii If $x \in \mathbf{S}$ is even but $3 \nmid x$ then $\psi(x) < \left( \half+\frac14+\frac15 \right) x < x$
    and $\psi(x)$ is still not a multiple of $3$.
    So if any $a_i$ is not divisible by $3$ the sequence is again strictly decreasing.
    \qedhere
  \end{itemize}
\end{proof}

On the other hand, if $x$ is a multiple of $6$, we have the following formula for $\psi(x)$:
\[ \psi(x) = \begin{cases}
  \frac{13}{12}x & 4 \mid x \\
  \frac{31}{30}x & 4 \nmid x \text{ but } 5 \mid x \\
  x & 4 \nmid x \text{ and } 5 \nmid x.
  \end{cases} \]
Looking back on our sequence of $a_i$ (which are all multiples of $6$),
the center case cannot happen with our $a_i$, because $\frac{31}{30}x$
is odd when $x \equiv 2 \pmod 4$.
Hence in actuality
\[ a_{n+1} = \frac{13}{12} a_n \quad\text{or}\quad a_{n+1} = a_n \]
for every $n$.

Let $T$ be the smallest index such that $a_T = a_{T+1} = a_{T+2} = \dotsb$
(it must exist because we cannot multiply by $\frac{13}{12}$ forever).
Then we can exactly describe the sequence by
\[ a_n = a_1 \cdot \left( \frac{13}{12} \right)^{\min(n,T)-1}. \]
Hence $a_1 = \left( \frac{12}{13} \right)^{T-1} a_T$,
and since $a_T$ is a multiple of $6$ not divisible by $4$ or $5$,
it follows $a_1$ has the required form.
\pagebreak

\subsection{IMO 2025/5, proposed by Massimiliano Foschi, Leonardo Franchi (ITA)}
\textsl{Available online at \url{https://aops.com/community/p35341177}.}
\begin{mdframed}[style=mdpurplebox,frametitle={Problem statement}]
Alice and Bazza are playing the \emph{inekoalaty game},
a two‑player game whose rules depend on a positive real number $\lambda$ which is known to both players.
On the $n$th turn of the game (starting with $n=1$) the following happens:
\begin{itemize}
  \ii If $n$ is odd, Alice chooses a nonnegative real number $x_n$ such that
  \[ x_1 + x_2 + \cdots + x_n \le \lambda n.  \]
  \ii If $n$ is even, Bazza chooses a nonnegative real number $x_n$ such that
  \[ x_1^2 + x_2^2 + \cdots + x_n^2 \le n.  \]
\end{itemize}
If a player cannot choose a suitable $x_n$, the game ends and the other player wins.
If the game goes on forever, neither player wins.
All chosen numbers are known to both players.

Determine all values of $\lambda$ for which Alice has a winning strategy
and all those for which Bazza has a winning strategy.
\end{mdframed}
The answer is that Alice has a winning strategy for $\lambda > 1/\sqrt2$,
and Bazza has a winning strategy for $\lambda < 1/\sqrt2$.
(Neither player can guarantee winning for $\lambda = 1/\sqrt2$.)

We divide the proof into two parts.

\paragraph{Alice's strategy when $\lambda \ge 1/\sqrt2$.}
Consider the strategy where Alice always plays $x_{2i+1} = 0$ for $i=0, \dots, k-1$.

In this situation, when $n = 2k+1$ we have
\begin{align*}
  \sum_1^{2k} x_i &= 0 + x_2 + 0 + x_4 + \dots + 0 + x_{2k}  \\
  &\le k \cdot \sqrt{\frac{x_2^2 + \dots + x_{2k}^2}{k}}
  = \sqrt{2} \cdot k < \lambda \cdot (2k+1)
\end{align*}
and so the choices for $x_{2k+1}$ are
\[ x_{2k+1} \in [0, \lambda \cdot (2k+1) - \sqrt{2} k] \]
which is nonempty.
Hence Alice can't ever lose with this strategy.

But suppose further $\lambda > \frac{1}{\sqrt 2}$; we show Alice can win.
Choose $k$ large enough that
\[ \sqrt{2} \cdot k < \lambda \cdot (2k+1) - \sqrt{2k+2}. \]
Then on the $(2k+1)$st turn, Alice can (after playing $0$ on all earlier turns)
play a number greater than $\sqrt{2k+2}$ and cause Bazza to lose.

\paragraph{Bazza strategy when $\lambda \le 1/\sqrt2$.}
Consider the strategy where Bazza always plays $x_{2i+2} = \sqrt{2 - x_{2i+1}^2}$
for all $i = 0, \dots, k-1$ (i.e.\ that is, the largest possible value Bazza can play).

To analyze Bazza's choices on each of his turns, we first need to estimate $x_{2k+1}$.
We do this by writing
\begin{align*}
  \lambda \cdot (2k+1) &\ge x_1 + x_2 + \dots + x_{2k+1} \\
  &= \left(x_1 + \sqrt{2-x_1^2}\right) + \left(x_3 + \sqrt{2-x_3^2}\right) \\
  &\qquad+ \dots + \left(x_{2k-1} + \sqrt{2-x_{2k}^2}\right) + x_{2k+1} \\
  &\ge {\underbrace{\sqrt{2} + \dots + \sqrt{2}}_{k \text{ times}}} + x_{2k+1}
  = \sqrt 2 \cdot k + x_{2k+1}
\end{align*}
where we have used the fact that $t + \sqrt{2-t^2} \ge 2$ for all $t \ge 0$.
This means that
\[ x_{2k+1} \le \lambda \cdot (2k+1) - \sqrt 2 k < \sqrt 2. \]
And $x_1^2 + x_2^2 + \dots + x_{2k}^2 + x_{2k+1}^2 = (2 + \dots + 2) + x_{2k+1}^2$,
Bazza can indeed choose $x_{2k+2} = \sqrt{2-x_{2k+1}^2}$ and always has a move.

But suppose further $\lambda < 1/\sqrt2$.
Then the above calculation \emph{also} shows that Alice couldn't
have made a valid choice for large enough $k$,
since $\lambda \cdot (2k+1) - \sqrt 2 k < 0$ for large $k$.

\begin{remark*}
  In the strategies above,
  we saw that Alice prefers to always play $0$
  and Bazza prefers to always play as large as possible.
  One could consider what happens in the opposite case:
  \begin{itemize}
    \ii If Alice tries to always play the largest number possible,
    her strategy still wins for $\lambda > 1$.
    \ii If Bazza tries to always play $0$,
    Alice can win no matter the value for $\lambda > 0$.
  \end{itemize}
\end{remark*}
\pagebreak

\subsection{IMO 2025/6, proposed by Zhao Yu Ma and David Lin Kewei (SGP)}
\textsl{Available online at \url{https://aops.com/community/p35341197}.}
\begin{mdframed}[style=mdpurplebox,frametitle={Problem statement}]
Consider a $2025 \times 2025$ grid of unit squares.
Matilda wishes to place on the grid some rectangular tiles, possibly of different sizes,
such that each side of every tile lies on a grid line
and every unit square is covered by at most one tile.

Determine the minimum number of tiles Matilda needs to place
so that each row and each column of the grid has exactly one unit square
that is not covered by any tile.
\end{mdframed}
The answer is $2112 = 2025 + 2 \cdot 45 - 3$.
In general, the answer turns out to be $\left\lceil n + 2 \sqrt n - 3 \right\rceil$,
but when $n$ is not a perfect square the solution is more complicated.

\begin{remark*}
  The 2017 Romanian Masters in Math asked the same problem where the tiles
  are replaced by \emph{sticks}, i.e.\ $1 \times k$ tiles.
  The answer to that problem is completely different and as far as I know
  there is no connection at all to the present IMO problem.
\end{remark*}

\paragraph{Construction.}
We show a general construction when $n = k^2$ illustrated below for $k=5$;
it generalizes readily.
There are a total of $(k-1)^2$ tiles which are $k \times k$ squares and another
$4(k-1)$ tiles on the boundary, giving a total of
\[ (k-1)^2 + 4(k-1) = k^2 + 2k - 3 \]
tiles as promised.

\begin{center}
\begin{asy}
  unitsize(0.4cm);
  int n = 5;
  int N = n*n;
  path outer = box((0,0),(N,N));
  pair P(int i, int j) {
    return (n*i+j,n*j-i+n-1);
  }
  for (int i=0; i<N; ++i) {
    draw((0,i)--(N,i), dotted);
    draw((i,0)--(i,N), dotted);
  }
  for (int i=-n; i<2*n; ++i) {
  for (int j=-n; j<2*n; ++j) {
    fill(shift(P(i,j))*unitsquare);
    filldraw(shift(P(i,j)+(1,0))*scale(n)*unitsquare, opacity(0.2)+yellow, red+1.2);
  }
  }
  clip(outer);
  draw(outer, blue+1.8);
\end{asy}
\end{center}

\begin{remark*}
  Ironically, the construction obtaining the answer
  in the floor pattern at Sunshine Coast airport,
  the closest airport to the site of the exam.
  See \href{https://i0.wp.com/havewheelchairwilltravel.net/wp-content/uploads/2019/10/IMG_6815.jpg}{this image}
  or \href{https://stea.com.au/wp-content/uploads/2020/12/sunshine-coast-airport-design-stea-architectural-projects-20.jpg}{this image}.
\end{remark*}

\paragraph{Bound.}
There are several approaches (all hard), but
the shortest proof seems to be the following one that exploits
the Erd\"{o}s-Szekeres theorem; it is Solution 6 in the shortlist.
The theorem being quoted is:
\begin{theorem*}
  [Erd\"{o}s-Szekeres]
  Let $n \ge 1$ be an integer.
  Given a permutation of $(1, \dots, n)$,
  if a \emph{longest increasing subsequence (LIS)} has length $a$ and
  a \emph{longest decreasing subsequence (LDS)} has length $b$, then $ab \ge n$.
\end{theorem*}
This is a stronger version of the theorem compared to another version
which instead just asserts that $\max(a,b) \ge \sqrt n$.

To apply this to the present problem, take the $n$ uncovered squares
which we henceforth call ``black'' as a permutation.
Then consider both an LIS of length $a$ and an LDS of length $b$.
We do the following artistic illustration:
\begin{itemize}
  \ii Draw the LIS as a broken line,
  then connect it to the southwest and northwest corner of the board.

  \ii Similarly, draw the LDS as a broken line,
  then connect it to the northwest and southeast corner of the board.

  \ii These two steps partition the board into four quadrants,
  which we call north, east, south, west.

  \ii For each black cell in the north quadrant,
  write an $\mathbf{N}$ in the cell above it
  (for the cell in the first row, this will be off the board).
  Do the same for $\mathbf{E}$ (east), $\mathbf{S}$ (south), $\mathbf{W}$ (west).

  \ii Some black cells are in multiple quadrants (i.e.\ part of the LIS/LDS).
  Write all letters in that case.
\end{itemize}
The figure below shows two examples of the process,
each for a board with $n=9$, for two choices of LIS and LDS.
The cells in the LIS and LDS have been marked with green circles,
and the boundaries of the quadrants are drawn in green lines.
In the left example, the LIS and LDS have a black square in common
(that cell has all four directions labeled).
In the right example, the LIS and do not have common squares.

\begin{center}
\begin{asy}
  unitsize(0.65cm);
  int n = 3;
  int N = n*n;
  path outer = box((0,0),(N,N));
  pair P(int i, int j) {
    return (n*i+j,n*j-i+n-1);
  }
  pair Q(int i, int j) {
    return P(i,j)+(0.5,0.5);
  }
  for (int i=0; i<N; ++i) {
    draw((0,i)--(N,i), dotted);
    draw((i,0)--(i,N), dotted);
  }
  for (int i=-n; i<2*n; ++i) {
  for (int j=-n; j<2*n; ++j) {
    fill(shift(P(i,j))*unitsquare);
    filldraw(shift(P(i,j)+(1,0))*scale(n)*unitsquare, opacity(0.2)+yellow, red+1.2);
  }
  }
  clip(outer);
  draw(outer, blue+1.8);
  draw((0,0)--Q(0,0)--Q(2,1)--Q(2,2)--(N,N), deepgreen+1.5);
  draw((0,N)--Q(0,2)--Q(1,2)--Q(2,1)--(N,0), deepgreen+1.5);
  real r = 0.4;
  filldraw(circle(Q(0,0), r), lightgreen, deepgreen+1.5);
  filldraw(circle(Q(2,1), r), lightgreen, deepgreen+1.5);
  filldraw(circle(Q(2,2), r), lightgreen, deepgreen+1.5);
  filldraw(circle(Q(0,2), r), lightgreen, deepgreen+1.5);
  filldraw(circle(Q(1,2), r), lightgreen, deepgreen+1.5);

  label("$\mathbf{N}$", (2.5,9.5));
  label("$\mathbf{N}$", (5.5,8.5));
  label("$\mathbf{N}$", (8.5,7.5));
  label("$\mathbf{N}$", (7.5,4.5));

  label("$\mathbf{W}$", (1.5,8.5));
  label("$\mathbf{W}$", (0.5,5.5));
  label("$\mathbf{W}$", (-0.5,2.5));
  label("$\mathbf{W}$", (4.5,7.5));
  label("$\mathbf{W}$", (3.5,4.5));
  label("$\mathbf{W}$", (6.5,3.5));

  label("$\mathbf{E}$", (9.5,6.5));
  label("$\mathbf{E}$", (8.5,3.5));

  label("$\mathbf{S}$", (7.5,2.5));
  label("$\mathbf{S}$", (0.5,1.5));
  label("$\mathbf{S}$", (3.5,0.5));
  label("$\mathbf{S}$", (6.5,-0.5));
\end{asy}
\qquad
\begin{asy}
  unitsize(0.65cm);
  int n = 3;
  int N = n*n;
  path outer = box((0,0),(N,N));
  pair P(int i, int j) {
    return (n*i+j,n*j-i+n-1);
  }
  pair Q(int i, int j) {
    return P(i,j)+(0.5,0.5);
  }
  for (int i=0; i<N; ++i) {
    draw((0,i)--(N,i), dotted);
    draw((i,0)--(i,N), dotted);
  }
  for (int i=-n; i<2*n; ++i) {
  for (int j=-n; j<2*n; ++j) {
    fill(shift(P(i,j))*unitsquare);
    filldraw(shift(P(i,j)+(1,0))*scale(n)*unitsquare, opacity(0.2)+yellow, red+1.2);
  }
  }
  clip(outer);
  draw(outer, blue+1.8);
  draw((0,0)--Q(0,0)--Q(0,1)--Q(2,2)--(N,N), deepgreen+1.5);
  draw((0,N)--Q(0,2)--Q(1,1)--Q(2,1)--(N,0), deepgreen+1.5);
  real r = 0.4;
  filldraw(circle(Q(0,0), r), lightgreen, deepgreen+1.5);
  filldraw(circle(Q(0,1), r), lightgreen, deepgreen+1.5);
  filldraw(circle(Q(2,2), r), lightgreen, deepgreen+1.5);
  filldraw(circle(Q(0,2), r), lightgreen, deepgreen+1.5);
  filldraw(circle(Q(1,1), r), lightgreen, deepgreen+1.5);
  filldraw(circle(Q(2,1), r), lightgreen, deepgreen+1.5);

  label("$\mathbf{N}$", (2.5,9.5));
  label("$\mathbf{N}$", (5.5,8.5));
  label("$\mathbf{N}$", (8.5,7.5));

  label("$\mathbf{W}$", (1.5,8.5));
  label("$\mathbf{W}$", (0.5,5.5));
  label("$\mathbf{W}$", (-0.5,2.5));

  label("$\mathbf{E}$", (9.5,6.5));
  label("$\mathbf{E}$", (8.5,3.5));
  label("$\mathbf{E}$", (5.5,4.5));

  label("$\mathbf{S}$", (7.5,2.5));
  label("$\mathbf{S}$", (1.5,4.5));
  label("$\mathbf{S}$", (0.5,1.5));
  label("$\mathbf{S}$", (3.5,0.5));
  label("$\mathbf{S}$", (4.5,3.5));
  label("$\mathbf{S}$", (6.5,-0.5));
\end{asy}
\end{center}

We observe that:
\begin{claim*}
  In this algorithm, the total number of letters written is exactly
  \[
    C \coloneq \begin{cases}
      n + a + b + 1 & \text{if the LIS and LDS intersect} \\
      n + a + b & \text{otherwise}.
    \end{cases}
  \]
\end{claim*}
\begin{proof}
  This is obvious. Each black square contributes at least one letter.
  Each black square on exactly one of the LIS and LDS contributes one extra letter.
  And a black square on both contributes $4$ letters instead of $1+1+1$.
\end{proof}
Note by AM-GM we have $a + b \ge 2 \sqrt{ab} \ge 2 \sqrt n$,
so we have a bound of
\[ C \ge n + 2 \sqrt n + \eps \quad\text{where}\quad
  \eps \coloneq \begin{cases}
    1 & \text{if the LIS and LDS intersect} \\
    0 & \text{otherwise}.
  \end{cases} \]

To relate $C$ to the number of tiles,
the critical claim is the following, which is by construction:
\begin{claim*}
  None of Matilda's tiles can have more than one letter written in any cell.
\end{claim*}
\begin{proof}
  This follows from the construction.
\end{proof}

We split into two cases based on $\eps$.
\begin{itemize}
  \ii When $\eps = 1$, at most four letters go off the grid
  (one for each direction), the number of tiles is at least $C - 4 \ge n + 2 \sqrt n - 3$.
  \ii Suppose $\eps = 0$.
  Then $C - 4 \ge n + 2 \sqrt n - 4$.
  However, we make the additional observation here that the tile
  where the LIS and LDS meet has no letters on it either;
  hence there are at least $1 + (C-4) \ge n + 2 \sqrt n - 3$ tiles.
\end{itemize}

\begin{remark*}
  \href{https://aops.com/community/p35341804}{USJL} mentions that when $n = ab$,
  it is in fact always possible to guarantee $\eps = 1$.
  Moreover, when $a \neq b$, the AM-GM inequality is strict.
  This gives a way to avoid the additional observation needed for $\eps = 0$ above.
\end{remark*}
\pagebreak


\end{document}
