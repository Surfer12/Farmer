% SPDX-License-Identifier: LicenseRef-Internal-Use-Only
% Copyright (c) 2025 Jumping Quail Solutions
% Classification: Confidential — Internal Use Only
\documentclass[12pt,a4paper]{article}
\usepackage[utf8]{inputenc}
\usepackage{amsmath,amssymb,amsthm}
\usepackage{booktabs}
\usepackage{array}
\usepackage{geometry}
\usepackage{xcolor}
\usepackage{graphicx}
\usepackage{hyperref}
\usepackage{fancyhdr}
\usepackage{enumitem}
\usepackage{listings}
\usepackage{csquotes}

\geometry{margin=1in}

\pagestyle{fancy}
\fancyhf{}
\lhead{Epistemic Confidence Model (Ψ)}
\rhead{Submission Draft}
\lfoot{Confidential — Internal Use Only}
\rfoot{\thepage}

\hypersetup{
  colorlinks=true,
  linkcolor=blue,
  urlcolor=blue,
  citecolor=blue
}

\title{Ψ: A Monotone, Auditable Epistemic Confidence Model with Safe Uplift and Governance\vspace{4pt}\\\large 2025 Computational Analysis, Operational Properties, and Licensing Guidance}
\author{Jumping Quail Solutions\\\small \textcopyright{} 2025 Jumping Quail Solutions. All rights reserved.}
\date{\today}

\lstdefinestyle{code}{
  basicstyle=\ttfamily\small,
  breaklines=true,
  frame=single,
  backgroundcolor=\color{gray!5},
  rulecolor=\color{gray!50},
  showstringspaces=false
}

\begin{document}

\maketitle

\begin{abstract}
We present Ψ, a single, bounded decision score that blends symbolic evidence with canonical sources, discounts authority and verifiability risks via an exponential penalty, and applies a calibrated posterior uplift. Ψ is designed for fast, auditable operations with predictable sensitivity. We formalize the core model, report computational behavior on representative scenarios, and state three operational invariants: gauge freedom, threshold transfer, and sensitivity invariants. We also show how Ψ composes with rules and MCDA for discrete choices. An appendix provides internal vs. public licensing guidance, a paste-ready internal-use license, notices, and SPDX headers.
\end{abstract}

\section{Mathematical Framework}
Let $S$ denote symbolic reasoning strength and $N$ denote canonical (external) evidence. The hybrid evidence score is
\begin{equation}
O(\alpha) = \alpha S + (1-\alpha) N, \quad \alpha\in[0,1].
\end{equation}
Risk factors enter multiplicatively through an exponential penalty
\begin{equation}
\mathrm{pen} = \exp\big(-[\lambda_1 R_a + \lambda_2 R_v]\big), \quad \lambda_1,\lambda_2\ge 0,
\end{equation}
where $R_a$ is authority risk and $R_v$ is verifiability risk. Expert/canonical uplift is captured with a capped posterior scaling
\begin{equation}
\mathrm{post}(\beta) = \min\{\beta \cdot P(H\mid E), 1\}, \quad \beta>0.
\end{equation}
The final decision score is
\begin{equation}
\boxed{\Psi = O(\alpha)\, \mathrm{pen}\, \mathrm{post}(\beta)}\ \in (0,1].
\end{equation}
Sensitivity to allocation is bounded and predictable:
\begin{equation}
\frac{\partial \Psi}{\partial \alpha} = (S-N)\, \mathrm{pen}\, \mathrm{post}(\beta).
\end{equation}

\section{Computational Behavior}
The model exhibits consistent, auditable responses across common regimes. Illustrative rows show the monotone effect of canonical dominance ($N>S$) and bounded uplift.
\begin{center}
\begin{tabular}{@{}cccccccc@{}}
\toprule
$\alpha$ & Hybrid & Penalty & Posterior & $\Psi$ & $\partial\Psi/\partial\alpha$ & Confidence & Label \\
\midrule
0.120 & 0.9256 & 0.8976 & 1.0000 & 0.831 & $-0.3321$ & 0.93 & Primitive/Evidence-Grounded \\
0.150 & 0.9145 & 0.8976 & 1.0000 & 0.821 & $-0.3321$ & 0.93 & Primitive/Evidence-Grounded \\
\midrule
0.170 & 0.8407 & 0.7965 & 0.9450 & 0.633 & $-0.2183$ & 0.85 & Interpretive/Contextual \\
0.150 & 0.8550 & 0.7965 & 0.9450 & 0.644 & $-0.2258$ & 0.85 & Interpretive/Contextual \\
\bottomrule
\end{tabular}
\end{center}

\section{Operational Properties}
\subsection{Gauge Freedom}
Renaming or trivially reparameterizing knobs does not change behavior. If two parameterizations compute the same $O(\alpha)$, penalty, and uplift, then they induce the same $\Psi$. Intuition: swap labels $\alpha\leftrightarrow \alpha_g$, $\lambda_i\leftrightarrow \lambda_{ig}$, $\beta\leftrightarrow\beta_g$; outputs and rankings are identical.

\subsection{Threshold Transfer}
When uplift changes from $\beta$ to $\beta'$, you can rescale the decision threshold to preserve accept/reject decisions (away from the hard cap):
\begin{equation}
\tau' = \tau\cdot (\beta/\beta').
\end{equation}
Example: if $\beta:1.2\to 1.5$ and $\tau=0.80$, then $\tau'=0.80\cdot(1.2/1.5)=0.64$, keeping decisions unchanged in the sub-cap region.

\subsection{Sensitivity Invariants}
Directions of change do not flip under trivial reparameterizations or threshold transfers. If $N>S$, decreasing $\alpha$ increases $\Psi$; increasing any risk reduces $\Psi$; changing $\beta$ scales sensitivities without changing signs or zero sets.

\section{Policy Without a Hard Cap}
You can remove the hard cap on \emph{belief} while keeping \emph{policy} bounded and auditable.
\begin{itemize}[leftmargin=*]
  \item \textbf{Dual-channel (recommended):} Raw belief $R = \beta\,O\,\mathrm{pen}$. Policy score $\Psi = \min\{R,1\}$ or $\Psi = 1-\exp(-R)$. Gate actions on $\Psi$; use $R$ for strength.
  \item \textbf{Uncapped with raw thresholds:} Act iff $R\ge \tau_{\text{raw}}$. Threshold transfer: if $\beta\to\beta'$, set $\tau' = \tau_{\text{raw}}\cdot(\beta'/\beta)$ to preserve decisions (previously sub-cap).
  \item \textbf{Soft cap:} $\Psi = 1-\exp(-\gamma R)$ for $\gamma>0$; strictly increasing with diminishing returns.
\end{itemize}
Monotonicity is preserved. Sensitivity to $\beta$ grows linearly in $R$; mitigate via change caps on $\beta$ and versioned thresholds.

\section{Composition with Rules and MCDA}
For discrete choices, filter by rules first, score options with $\Psi(a)$, then apply a monotone MCDA aggregator $M$ over a criteria vector $c(a)=[\Psi(a),\ \text{cost}(a),\ \text{value}(a),\ldots]$. If $M$ is strictly increasing in the $\Psi$ coordinate, then gauge freedom and sub-cap threshold transfer preserve rankings induced by $M$ when other criteria are fixed.

\section{Governance}
\begin{itemize}[leftmargin=*]
  \item \textbf{Observable triggers:} $\alpha$ drops with canonical artifacts; $R_a,R_v$ fall with official URLs; $\beta$ rises with certification.
  \item \textbf{Audit trail:} Sources \textrightarrow{} Hybrid \textrightarrow{} Penalty \textrightarrow{} Posterior stages report confidence components.
  \item \textbf{Guardrails:} Nonnegative penalties and bounded policy prevent runaway uplift; version thresholds and parameter changes.
\end{itemize}

\section{Conclusions}
Ψ provides a single, monotone, bounded decision score with transparent levers, stable sensitivities, and safe uplift. It composes cleanly with rules and MCDA and supports repeatable operations with clear governance.

\section*{Acknowledgments and Notes}
This document is provided for operational clarity and auditability. It is not legal advice.

\appendix
\section{Licensing and Distribution Guidance}\label{app:licensing}
\subsection*{Internal vs. Public Licensing (Ψ Work)}
\textbf{Internal (default)}: Use the Internal-Use-Only license below for private collaboration, evaluation, and development.\newline
\textbf{Public (open) options}: For public release of selected artifacts, choose either CC0 (dedication to the public domain) for documents/datasets or GPL-3.0-only for source code derivatives, per project policy and NDA constraints.

\subsection*{Paste-Ready Internal-Use License}
\noindent\textbf{LicenseRef-Internal-Use-Only v1.0}
\begin{lstlisting}[style=code]
Grant: The Owner grants Collaborators a non-exclusive, non-transferable, revocable, royalty-free license to use, reproduce, and modify this work solely for internal evaluation and collaboration on this project.

Restrictions: No distribution outside the collaboration; no sublicensing; retain copyright and license notices; treat non-public materials as confidential.

Contributions: Each Contributor grants the Owner a non-exclusive, perpetual, irrevocable license to use and relicense their contributions for the project. Contributors represent they have the right to contribute.

IP/Patents: No patent license is granted beyond what is necessary to use the contribution internally. No trademark rights.

Termination: This license terminates automatically upon breach or upon written notice. On termination, stop all use except archival records.

No Warranty / No Liability: The work is provided “AS IS” without warranties. To the extent permitted by law, the Owner disclaims liability.

Governing Law: [Your jurisdiction].
\end{lstlisting}

\subsection*{Public Licensing Options}
\begin{itemize}[leftmargin=*]
  \item \textbf{CC0 1.0 (recommended for text/figures/datasets):} Dedicates content to the public domain. Add a clear notice in the README and file headers.
  \item \textbf{GPL-3.0-only (recommended for code):} Ensures derivatives remain open under copyleft. Provide a top-level LICENSE file and SPDX headers in every source file.
\end{itemize}

\subsection*{Notices}
\begin{itemize}[leftmargin=*]
  \item \textbf{Copyright:} \textcopyright{} 2025 Jumping Quail Solutions. All rights reserved.
  \item \textbf{Classification:} Confidential — Internal Use Only.
  \item \textbf{Policy:} GPLv3 for public code releases where approved; otherwise internal-use only per above and applicable NDAs.
\end{itemize}

\subsection*{SPDX File Headers (Examples)}
Include one header at the top of each file using the appropriate comment syntax.

\noindent\textbf{Java}
\begin{lstlisting}[style=code]
/*
 * SPDX-License-Identifier: LicenseRef-Internal-Use-Only
 * Copyright (c) 2025 Jumping Quail Solutions
 */
\end{lstlisting}

\noindent\textbf{Swift}
\begin{lstlisting}[style=code]
// SPDX-License-Identifier: LicenseRef-Internal-Use-Only
// Copyright (c) 2025 Jumping Quail Solutions
\end{lstlisting}

\noindent\textbf{LaTeX}
\begin{lstlisting}[style=code]
% SPDX-License-Identifier: LicenseRef-Internal-Use-Only
% Copyright (c) 2025 Jumping Quail Solutions
\end{lstlisting}

\noindent\textbf{Markdown}
\begin{lstlisting}[style=code]
<!-- SPDX-License-Identifier: LicenseRef-Internal-Use-Only -->
<!-- Copyright (c) 2025 Jumping Quail Solutions -->
\end{lstlisting}

\noindent\textbf{Python}
\begin{lstlisting}[style=code]
# SPDX-License-Identifier: LicenseRef-Internal-Use-Only
# Copyright (c) 2025 Jumping Quail Solutions
\end{lstlisting}

\subsection*{Switching to Public Releases}
For selected files slated for public release:
\begin{itemize}[leftmargin=*]
  \item Replace the SPDX tag with \texttt{SPDX-License-Identifier: CC0-1.0} (text/data) or \texttt{SPDX-License-Identifier: GPL-3.0-only} (code).
  \item Add or update top-level LICENSE and a \texttt{NOTICE} file listing third-party attributions.
  \item Version and record the threshold transfer (if any) and parameter changes tied to the release.
\end{itemize}

\section*{References}
% If a bibliography is desired, enable the following two lines and run BibTeX.
% \bibliographystyle{abbrv}
% \bibliography{references,citations}

\end{document}