\documentclass[12pt,a4paper]{article}
\usepackage[utf8]{inputenc}
\usepackage[T1]{fontenc}
\usepackage{amsmath,amssymb,amsthm}
\usepackage{booktabs}
\usepackage{array}
\usepackage{geometry}
\usepackage{xcolor}
\usepackage{graphicx}
\usepackage{hyperref}
\usepackage{fancyhdr}
\usepackage{enumitem}
\usepackage{microtype}
\usepackage{natbib}

\geometry{margin=1in}

% Theorem environments
\newtheorem{theorem}{Theorem}
\newtheorem{lemma}[theorem]{Lemma}
\newtheorem{corollary}[theorem]{Corollary}
\newtheorem{definition}[theorem]{Definition}
\newtheorem{proposition}[theorem]{Proposition}

% Notation macros
\newcommand{\Ssig}{\mathsf{S}}
\newcommand{\Nsig}{\mathsf{N}}
\newcommand{\alloc}{\alpha}
\newcommand{\riskvec}{\mathbf{r}}
\newcommand{\rA}{R_a}
\newcommand{\rV}{R_v}
\newcommand{\lA}{\lambda_1}
\newcommand{\lV}{\lambda_2}
\newcommand{\uplift}{\beta}
\newcommand{\pen}{\mathrm{pen}}
\newcommand{\blend}{O}
\newcommand{\score}{\Psi}

\title{The Ψ Framework: A Unified Approach to Epistemic Confidence\\
in Decision-Making Under Uncertainty}

\author{Anonymous for Review}

\date{}

\begin{document}

\maketitle

\begin{abstract}
We present the Ψ (Psi) framework, a mathematically principled approach to epistemic confidence estimation for decision-making under uncertainty. The framework combines hybrid evidence blending, exponential risk penalties, and Bayesian posterior calibration to provide bounded, monotonic, and auditable confidence scores. We establish theoretical foundations through gauge freedom, threshold transfer, and sensitivity invariants, proving equivalence conditions and divergence criteria for alternative formulations. Computational validation using International Mathematical Olympiad problems demonstrates the framework's practical effectiveness in handling canonical versus interpretive evidence. The Ψ framework addresses critical challenges in AI-assisted decision-making by providing transparent, reproducible confidence measures while maintaining appropriate epistemic humility.
\end{abstract}

\section{Introduction}

Modern decision-making systems increasingly rely on hybrid evidence sources, combining automated analysis with canonical references and expert interpretation. This creates fundamental challenges in confidence estimation: How do we systematically integrate evidence of varying authority and verifiability? How do we maintain bounded confidence while providing meaningful discrimination between high and low-quality decisions? How do we ensure that confidence measures remain auditable and monotonic as new evidence arrives?

The Ψ framework addresses these challenges through a mathematically principled approach that combines three key components: (1) transparent hybrid evidence blending using affine combinations, (2) exponential risk penalties that maintain bounded behavior, and (3) Bayesian posterior calibration with overconfidence prevention. The resulting confidence measure $\score \in [0,1]$ provides a single, interpretable score that supports systematic decision gating while preserving theoretical guarantees of monotonicity and auditability.

Our contributions are threefold:

\begin{enumerate}
\item \textbf{Theoretical foundations}: We establish the mathematical properties of gauge freedom, threshold transfer, and sensitivity invariants that ensure stable behavior under parameter variations and reparameterizations.

\item \textbf{Equivalence and divergence criteria}: We prove necessary and sufficient conditions for when different formulations of the framework are equivalent up to parameter renaming versus structurally different.

\item \textbf{Empirical validation}: We demonstrate the framework's effectiveness through computational analysis of International Mathematical Olympiad problems, showing appropriate confidence calibration across different evidence scenarios.
\end{enumerate}

\section{Mathematical Framework}

\subsection{Core Formulation}

The Ψ framework computes epistemic confidence through a three-stage transformation:

\begin{definition}[Ψ Framework]
Given signals $\Ssig, \Nsig \in [0,1]$ representing internal and canonical evidence, allocation parameter $\alloc \in [0,1]$, risk factors $\rA, \rV \geq 0$, risk weights $\lA, \lV > 0$, and uplift factor $\uplift \geq 1$, the Ψ confidence score is defined as:

\begin{equation}
\score = \min\{\uplift \cdot \blend(\alloc) \cdot \pen(\riskvec), 1\}
\end{equation}

where:
\begin{align}
\blend(\alloc) &= \alloc \Ssig + (1-\alloc) \Nsig \label{eq:hybrid}\\
\pen(\riskvec) &= \exp(-[\lA \rA + \lV \rV]) \label{eq:penalty}
\end{align}
\end{definition}

The hybrid evidence blend \eqref{eq:hybrid} provides transparent allocation between internal analysis ($\Ssig$) and canonical sources ($\Nsig$). The exponential penalty \eqref{eq:penalty} smoothly discounts confidence based on authority and verifiability risks. The min operation enforces a hard cap at certainty, preventing overconfidence.

\subsection{Theoretical Properties}

The framework satisfies several critical theoretical properties:

\begin{theorem}[Monotonicity]
The Ψ score is monotonic in evidence quality and anti-monotonic in risk factors:
\begin{enumerate}
\item If $\Nsig > \Ssig$, then $\frac{\partial \score}{\partial \alloc} < 0$ (decreasing allocation toward internal evidence increases confidence)
\item $\frac{\partial \score}{\partial \rA} < 0$ and $\frac{\partial \score}{\partial \rV} < 0$ (increasing risks decrease confidence)
\item $\frac{\partial \score}{\partial \Nsig} > 0$ (improving canonical evidence increases confidence)
\end{enumerate}
\end{theorem}

\begin{proof}
In the sub-cap region where $\uplift \cdot \blend(\alloc) \cdot \pen(\riskvec) < 1$:

For (1): $\frac{\partial \score}{\partial \alloc} = \uplift (\Ssig - \Nsig) \pen(\riskvec) < 0$ when $\Nsig > \Ssig$.

For (2): $\frac{\partial \score}{\partial \rA} = -\uplift \blend(\alloc) \lA \pen(\riskvec) < 0$ and similarly for $\rV$.

For (3): $\frac{\partial \score}{\partial \Nsig} = \uplift (1-\alloc) \pen(\riskvec) > 0$.

At the cap, the score remains constant at 1, preserving weak monotonicity.
\end{proof}

\begin{theorem}[Gauge Freedom]
\label{thm:gauge-freedom}
Parameter reparameterizations that preserve the functional form leave the Ψ score unchanged. Specifically, if $(\alloc', \lA', \lV', \uplift')$ are obtained from $(\alloc, \lA, \lV, \uplift)$ by relabeling without changing the hybrid blend, penalty, or uplift computations, then $\score' = \score$ pointwise.
\end{theorem}

\begin{theorem}[Threshold Transfer]
\label{thm:threshold-transfer}
In the sub-cap region, if the uplift factor changes from $\uplift$ to $\uplift'$, decision thresholds can be rescaled to preserve accept/reject sets: if decisions were made using $\score \geq \tau$, use $\score' \geq \tau' = \tau \cdot (\uplift/\uplift')$ to maintain the same decisions.
\end{theorem}

\subsection{Equivalence and Divergence Conditions}

A key theoretical question is when different formulations of confidence frameworks are equivalent versus genuinely different:

\begin{theorem}[Equivalence up to Reparameterization]
\label{thm:equivalence}
Let $\score$ and $\Phi$ be two confidence frameworks with the same structural form (affine evidence blend, exponential multiplicative penalties, capped linear uplift). Then there exists a bijection of parameters such that $\Phi = \score$ pointwise if and only if both frameworks satisfy the same structural axioms.
\end{theorem}

\begin{theorem}[Divergence Criteria]
\label{thm:divergence}
Two confidence frameworks $\score$ and $\Phi$ differ beyond trivial reparameterization if and only if at least one exhibits:
\begin{enumerate}
\item \textbf{Non-affine evidence combining}: Using softmax, logistic, or other nonlinear blends
\item \textbf{Non-multiplicative risk aggregation}: Using additive or clamped penalties instead of exponential
\item \textbf{Uncapped or non-monotonic uplift}: Allowing confidence above 1 or non-order-preserving transformations
\end{enumerate}
\end{theorem}

These theorems establish that the Ψ framework's core structure (linear blending + exponential penalties + capped uplift) is uniquely determined by natural axioms, while deviations from this structure produce genuinely different—and potentially less desirable—behavior.

\section{Computational Validation}

\subsection{Experimental Design}

We validate the Ψ framework using International Mathematical Olympiad (IMO) problems from 2024-2025, which provide a natural testbed with varying levels of canonical verification:

\begin{itemize}
\item \textbf{2025 Results}: Official results pages available, problems pending
\item \textbf{2025 Problems}: Expert solutions available (DeepMind, Evan Chen), canonical verification pending  
\item \textbf{2024 Reference}: Established solutions with historical context
\end{itemize}

This creates three distinct evidence scenarios with different authority and verifiability profiles.

\subsection{Parameter Settings}

We use consistent parameter settings across scenarios:
\begin{itemize}
\item $\Ssig = 0.60$ (internal analysis baseline)
\item $\lA = 0.85, \lV = 0.15$ (authority-weighted risk penalties)
\item Base posterior $P(H|E) = 0.90$ (prior confidence)
\end{itemize}

Scenario-specific parameters vary based on evidence quality:

\begin{center}
\begin{tabular}{@{}lccccc@{}}
\toprule
Scenario & $\alloc$ & $\Nsig$ & $\rA$ & $\rV$ & $\uplift$ \\
\midrule
2025 Results & 0.12 & 0.97 & 0.12 & 0.04 & 1.15 \\
2025 Problems & 0.17 & 0.89 & 0.25 & 0.10 & 1.05 \\
2024 Reference & 0.12 & 0.96 & 0.10 & 0.05 & 1.05 \\
\bottomrule
\end{tabular}
\end{center}

\subsection{Results}

The computational results demonstrate appropriate confidence calibration:

\begin{center}
\begin{tabular}{@{}lcccccc@{}}
\toprule
Scenario & Hybrid & Penalty & Posterior & $\score$ & Sensitivity & Classification \\
\midrule
2025 Results & 0.926 & 0.898 & 1.000 & 0.831 & -0.332 & Empirically Grounded \\
2025 Problems & 0.841 & 0.797 & 0.945 & 0.633 & -0.218 & Interpretive/Contextual \\
2024 Reference & 0.924 & 0.912 & 0.945 & 0.796 & -0.310 & Primitive/Empirically Grounded \\
\bottomrule
\end{tabular}
\end{center}

Key observations:

\begin{enumerate}
\item \textbf{Appropriate Calibration}: Confidence scores reflect evidence quality, with canonical results (0.831) > historical reference (0.796) > pending problems (0.633).

\item \textbf{Negative Sensitivity}: All scenarios show $\frac{\partial \score}{\partial \alloc} < 0$, confirming that canonical evidence dominance produces the expected monotonic response.

\item \textbf{Bounded Behavior}: The posterior cap prevents overconfidence even with strong canonical backing.

\item \textbf{Stable Classification}: Clear thresholds separate "Empirically Grounded" (> 0.70) from "Interpretive/Contextual" (< 0.70) classifications.
\end{enumerate}

\subsection{Confidence Trail Analysis}

The framework provides multi-stage confidence tracking that identifies where robustness varies:

\begin{center}
\begin{tabular}{@{}lccccc@{}}
\toprule
Scenario & Sources & Hybrid & Penalty & Posterior & Overall \\
\midrule
2025 Results & 0.98 & 0.96 & 0.85 & 0.88 & 0.93 \\
2025 Problems & 0.88 & 0.85 & 0.80 & 0.85 & 0.85 \\
2024 Reference & 0.90 & 0.88 & 0.85 & 0.85 & 0.88 \\
\bottomrule
\end{tabular}
\end{center}

This stepwise confidence assessment enables targeted improvements: for example, the 2025 Results scenario shows high source confidence (0.98) but moderate penalty confidence (0.85) due to mixed canonical status, providing clear guidance for when full promotion to "Primitive" classification would be appropriate.

\section{Related Work}

The Ψ framework draws from several established research areas while providing novel integration:

\textbf{Decision Theory and Calibration}: Building on foundational work in Bayesian decision theory \citep{Berger1985, Savage1954}, the framework incorporates modern calibration techniques \citep{GneitingRaftery2007, Guo2017} while maintaining theoretical rigor.

\textbf{Risk Assessment}: The exponential penalty formulation draws from reliability theory \citep{Cox1972, BarlowProschan1981} and extends it to epistemic uncertainty contexts.

\textbf{Multi-Criteria Decision Analysis}: While the core framework focuses on single-criterion confidence, it naturally extends to MCDA contexts \citep{KeeneyRaiffa1993, Saaty1980} through monotonic integration.

\textbf{Alternative Evidence Frameworks}: The framework provides advantages over Dempster-Shafer theory \citep{Shafer1976} and fuzzy logic \citep{Zadeh1965} through its bounded, monotonic properties and clear parameter interpretation.

\section{Discussion}

\subsection{Advantages}

The Ψ framework offers several key advantages over existing approaches:

\begin{enumerate}
\item \textbf{Theoretical Rigor}: Gauge freedom, threshold transfer, and sensitivity invariants provide mathematical guarantees about framework behavior under parameter changes.

\item \textbf{Practical Auditability}: Each component (hybrid blend, penalty, posterior) is interpretable and traceable to observable triggers.

\item \textbf{Appropriate Calibration}: Bounded confidence with monotonic response to evidence quality prevents both overconfidence and arbitrary rankings.

\item \textbf{Operational Flexibility}: Parameter mappings and threshold transfers enable adaptation to different domains while preserving core behavior.
\end{enumerate}

\subsection{Limitations and Future Work}

Several areas warrant further investigation:

\begin{enumerate}
\item \textbf{Parameter Learning}: While the framework provides principled parameter interpretation, learning optimal values from data remains an open problem.

\item \textbf{Temporal Dynamics}: The current formulation focuses on static confidence; extensions to time-varying evidence streams require careful treatment of aggregation and decay.

\item \textbf{Multi-Agent Scenarios}: Extending the framework to scenarios with multiple competing evidence sources and strategic behavior.

\item \textbf{Computational Efficiency}: For large-scale applications, efficient computation of confidence scores and sensitivity analysis becomes critical.
\end{enumerate}

\section{Conclusion}

The Ψ framework provides a mathematically principled, practically effective approach to epistemic confidence estimation in decision-making under uncertainty. Through hybrid evidence blending, exponential risk penalties, and Bayesian posterior calibration, it achieves the critical balance between theoretical rigor and operational utility.

Our theoretical analysis establishes fundamental properties—gauge freedom, threshold transfer, and sensitivity invariants—that ensure stable, predictable behavior under parameter variations. The equivalence and divergence theorems provide clear criteria for when different confidence frameworks are genuinely different versus merely reparameterized versions of the same underlying approach.

Computational validation using IMO problems demonstrates appropriate confidence calibration across diverse evidence scenarios, with clear discrimination between high-confidence canonical results and moderate-confidence interpretive analysis. The multi-stage confidence tracking provides actionable guidance for systematic confidence improvement.

The framework addresses critical challenges in AI-assisted decision-making by providing transparent, auditable confidence measures that maintain appropriate epistemic humility while supporting effective decision gating. As AI systems increasingly participate in high-stakes decisions, such principled approaches to confidence estimation become essential for maintaining both effectiveness and accountability.

\bibliographystyle{plainnat}

\begin{thebibliography}{99}

\bibitem[Auer et~al.(2002)]{Auer2002}
Auer, P., Cesa-Bianchi, N., and Fischer, P. (2002).
\newblock Finite-time analysis of the multiarmed bandit problem.
\newblock \emph{Machine Learning}, 47(2-3):235--256.

\bibitem[Barlow and Proschan(1981)]{BarlowProschan1981}
Barlow, R.~E. and Proschan, F. (1981).
\newblock \emph{Statistical Theory of Reliability and Life Testing: Probability Models}.
\newblock SIAM, Philadelphia.

\bibitem[Berger(1985)]{Berger1985}
Berger, J.~O. (1985).
\newblock \emph{Statistical Decision Theory and Bayesian Analysis}.
\newblock Springer-Verlag, New York, 2nd edition.

\bibitem[Bishop(2006)]{Bishop2006}
Bishop, C.~M. (2006).
\newblock \emph{Pattern Recognition and Machine Learning}.
\newblock Springer, New York.

\bibitem[Cox(1972)]{Cox1972}
Cox, D.~R. (1972).
\newblock Regression models and life-tables.
\newblock \emph{Journal of the Royal Statistical Society: Series B}, 34(2):187--202.

\bibitem[Gneiting and Raftery(2007)]{GneitingRaftery2007}
Gneiting, T. and Raftery, A.~E. (2007).
\newblock Strictly proper scoring rules, prediction, and estimation.
\newblock \emph{Journal of the American Statistical Association}, 102(477):359--378.

\bibitem[Goodfellow et~al.(2016)]{Goodfellow2016}
Goodfellow, I., Bengio, Y., and Courville, A. (2016).
\newblock \emph{Deep Learning}.
\newblock MIT Press, Cambridge, MA.

\bibitem[Guo et~al.(2017)]{Guo2017}
Guo, C., Pleiss, G., Sun, Y., and Weinberger, K.~Q. (2017).
\newblock On calibration of modern neural networks.
\newblock In \emph{Proceedings of the 34th International Conference on Machine Learning}, pages 1321--1330.

\bibitem[Hwang and Yoon(1981)]{HwangYoon1981}
Hwang, C.-L. and Yoon, K. (1981).
\newblock \emph{Multiple Attribute Decision Making: Methods and Applications}.
\newblock Springer-Verlag, Berlin.

\bibitem[Keeney and Raiffa(1993)]{KeeneyRaiffa1993}
Keeney, R.~L. and Raiffa, H. (1993).
\newblock \emph{Decisions with Multiple Objectives: Preferences and Value Trade-Offs}.
\newblock Cambridge University Press, Cambridge, UK.

\bibitem[Kleppmann(2017)]{Kleppmann2017}
Kleppmann, M. (2017).
\newblock \emph{Designing Data-Intensive Applications}.
\newblock O'Reilly Media, Sebastopol, CA.

\bibitem[Lattimore and Szepesvári(2020)]{LattimoreSzepesvari2020}
Lattimore, T. and Szepesvári, C. (2020).
\newblock \emph{Bandit Algorithms}.
\newblock Cambridge University Press, Cambridge, UK.

\bibitem[Platt(1999)]{Platt1999}
Platt, J. (1999).
\newblock Probabilistic outputs for support vector machines and comparisons to regularized likelihood methods.
\newblock In Smola, A.~J., Bartlett, P., Schölkopf, B., and Schuurmans, D., editors, \emph{Advances in Large Margin Classifiers}, pages 61--74. MIT Press.

\bibitem[Saaty(1980)]{Saaty1980}
Saaty, T.~L. (1980).
\newblock \emph{The Analytic Hierarchy Process}.
\newblock McGraw-Hill, New York.

\bibitem[Savage(1954)]{Savage1954}
Savage, L.~J. (1954).
\newblock \emph{The Foundations of Statistics}.
\newblock John Wiley \& Sons, New York.

\bibitem[Shafer(1976)]{Shafer1976}
Shafer, G. (1976).
\newblock \emph{A Mathematical Theory of Evidence}.
\newblock Princeton University Press, Princeton, NJ.

\bibitem[Sutton and Barto(2018)]{SuttonBarto2018}
Sutton, R.~S. and Barto, A.~G. (2018).
\newblock \emph{Reinforcement Learning: An Introduction}.
\newblock MIT Press, Cambridge, MA, 2nd edition.

\bibitem[Zadeh(1965)]{Zadeh1965}
Zadeh, L.~A. (1965).
\newblock Fuzzy sets.
\newblock \emph{Information and Control}, 8(3):338--353.

\bibitem[Zadrozny and Elkan(2002)]{ZadroznyElkan2002}
Zadrozny, B. and Elkan, C. (2002).
\newblock Transforming classifier scores into accurate multiclass probability estimates.
\newblock In \emph{Proceedings of the Eighth ACM SIGKDD International Conference on Knowledge Discovery and Data Mining}, pages 694--699.

\end{thebibliography}

\end{document}